%定義

%\documentclass[10pt,twocolumn]{jarticle} 
\documentclass[11pt,dvipdfmx]{jreport}

\usepackage{algorithmic, algorithm}
\usepackage{graphicx}
\usepackage{fancybox}
\usepackage{comment}
\usepackage{amsmath}
\usepackage{amssymb}
\usepackage{amsfonts}
\usepackage{euler}
%\usepackage{mathrsfs}
\usepackage{color}
\usepackage{ulem}
\usepackage{proof}
\usepackage[ND,SEQ,IMP]{prftree}
\usepackage[deluxe]{otf}


%\def\imp{\to}

%\pagestyle{empty}

%余白とか

%\setlength{\abovecaptionskip}{-10mm}
\setlength{\topmargin}{-3.0cm} 
\setlength{\textheight}{27.0cm} 
\setlength{\textwidth}{18.5cm}
\setlength{\oddsidemargin}{-1.3cm} 
\setlength{\columnsep}{.5cm}
\newcommand{\noin}{\noindent}
\catcode`@=\active \def@{\hspace{0.9bp}-\hspace{0.9bp}}
\newtheorem{dfn}{定義}[section]

%% \makeatletter
%% \renewcommand{\chapter}{\@startsection{chapter}{1}{\z@}
%% {1.5\Cvs \@plus.5\Cvs \@minus.2\Cvs}
%% {.5\Cvs \@plus.3\Cvs}
%% {\reset@font\Huge\mcfamily}}
%% \renewcommand{\section}{\@startsection{section}{1}{\z@}
%% {1.5\Cvs \@plus.5\Cvs \@minus.2\Cvs}
%% {.5\Cvs \@plus.3\Cvs}
%% {\reset@font\Large\mcfamily}}
%% \makeatother

%タイトル

\title{「情報科学における論理\footnote{小野寛晰, 日本評論社, 1994}」問題解答集(途中省略有り)}
\setcounter{footnote}{0}
\author{高田 篤司\thanks{神奈川大学理学部情報科学科} \and 原田 崇司\thanks{神奈川大学大学院理学研究科情報科学専攻 田中研究室}}
\date{\today}
\西暦

%タイトル作成

\begin{document}

\maketitle
%\thispagestyle{empty}

% 1.何について話すか(概要) => 2.具体例(図だけで十分) => 3.結論(一番伝えたい事)
%
% 結論に向けて,流れを意識して書く.
%
% 構文.主語,述語,目的語をはっきり書く.
%
% 受動態を使わない.
%
% 辞書,参考書,先生に伺うことより,適切な単語,表現で文章を作成する.
%
% 内容を削る.必要最小限
%
% コンマを列挙以外で2つ以上使う,``~の~の'',の如きは使わない.

% 

\noindent \textbf{問2.14} 
\par
\vspace{3mm}

\begin{enumerate}
  \renewcommand{\labelenumi}{\arabic{enumi}) }
\item
\begin{displaymath}
\setcounter{prfassumptioncounter}{0}
\prftree[r]{}
        {\prftree[r]{}
          {\prftree[r]{}
            {\prftree[r]{}
              {\prftree[r]{}
                {\prftree[r]{}
                  {\prftree[r]{}
                    { \, P(x) \to \, P(x) }
                    { \, P(x) \to \, P(x) , \,Q(x)}
                  }
                  { \to \, P(x) ,  P(x) \supset Q(x)}
                }
                { \to \forall x P(x) ,  P(x) \supset Q(x)}
              }
              { \to \forall x P(x) , \exists x ( P(x) \supset Q(x) )}
            }
            {\to \exists x ( P(x) \supset Q(x) ) , \forall x P(x)}
          }
          {\prftree[r]{}
            { \, Q(t) \to \, Q(t) }
            {\prftree[r]{}
              {\prftree[r]{}
                {\prftree[r]{}
                  { \,P(t) , \, Q(t) \to \, Q(t) )}
                  { \, Q(t) \to \, P(t) \supset Q(t) )}
                }
                { \, Q(t) \to \exists x ( P(x) \supset Q(x) )}
              }
              { \exists x Q(t) \to \exists x ( P(x) \supset Q(x) )}  
            }
          } 
          {\forall x P(x) \supset \exists Q(x) \to \exists x ( P(x) \supset Q(x) ) , \exists x ( P(x) \supset Q(x) )}
        }
        {\forall x P(x) \supset \exists Q(x) \to \exists x ( P(x) \supset Q(x) )}
\end{displaymath}\vspace{.2ex}
\item
  
\end{enumerate}

\appendix

\section{証明の書き方}
\begin{color}{blue}
\begin{itemize}
 \item 接続詞などに用いる用語を統一する(教科書を参考にする).
 \item 証明を書くときは,一行ずつ書いて改行する.
 \item サ変動詞を用いない.\sout{$\sim$として,$\sim$とする} $\ \Longrightarrow \ $ $\sim$と仮定する,$\sim$と置く,$\dots$となるような$\sim$をとる.
 \item 仮定が何で結論は何なのかを明示する.
 \item 問題文の情報を用いた場合は,問題文のどこを用いたのかを明示する.
 \item 推論する場合は,用いた根拠と用いた推論規則を明示する.
\end{itemize}
\end{color}


%% \begin{displaymath}
%% \setcounter{prfassumptioncounter}{0}
%% \prftree[r]{$\scriptstyle\supset\mathrm{I}$}
%%  {\prftree[r]{$\scriptstyle\land\mathrm{I}$}
  %% {\prftree[r]{$\scriptstyle\forall\mathrm{I}$}
  %%  {\prftree[r]{$\scriptstyle\land\mathrm{E1}$}
  %%   {\prftree[r]{$\scriptstyle\forall\mathrm{E}$}
  %%    {\prfboundedstyle=1\prfboundedassumption<assum:62>{\forall x \ (A \land B)}}
  %%    {A[z/x] \land B[z/x]}
  %%   }
  %%   {A[z/x]}
  %%  }
  %%  {\forall x\, A}
  %% }
  %% {\prftree[r]{$\scriptstyle\forall\mathrm{I}$}
  %%  {\prftree[r]{$\scriptstyle\land\mathrm{E2}$}
  %%   {\prftree[r]{$\scriptstyle\forall\mathrm{E}$}
  %%    {\prfboundedstyle=1\prfboundedassumption<assum:62>{\forall x \ (A \land B)}}
  %%    {A[z/x] \land B[z/x]}
  %%   }
  %%   {B[z/x]}
  %%  }
  %%  {\forall x\, B}
  %% }
%%   {\forall x \, A \land \forall x \, B}
%%  }
%% {\forall x \ (A \land B) \supset (\forall x \, A \land \forall x \, B)}
%% \end{displaymath}\vspace{.2ex}

\end{document}


%定義

%\documentclass[10pt,twocolumn]{jarticle} 
\documentclass[11pt,dvipdfmx]{jreport}

\usepackage{algorithmic, algorithm}
\usepackage{graphicx}
\usepackage{fancybox}
\usepackage{comment}
\usepackage{amsmath}
\usepackage{amssymb}
\usepackage{amsfonts}
\usepackage{euler}
%\usepackage{mathrsfs}
\usepackage{color}
\usepackage{ulem}
\usepackage{proof}
\usepackage[ND,SEQ,IMP]{prftree}
\usepackage[deluxe]{otf}


%\def\imp{\to}

%\pagestyle{empty}

%余白とか

%\setlength{\abovecaptionskip}{-10mm}
\setlength{\topmargin}{-3.0cm} 
\setlength{\textheight}{27.0cm} 
\setlength{\textwidth}{18.5cm}
\setlength{\oddsidemargin}{-1.3cm} 
\setlength{\columnsep}{.5cm}
\newcommand{\noin}{\noindent}
\catcode`@=\active \def@{\hspace{0.9bp}-\hspace{0.9bp}}
\newtheorem{dfn}{定義}[section]

%% \makeatletter
%% \renewcommand{\chapter}{\@startsection{chapter}{1}{\z@}
%% {1.5\Cvs \@plus.5\Cvs \@minus.2\Cvs}
%% {.5\Cvs \@plus.3\Cvs}
%% {\reset@font\Huge\mcfamily}}
%% \renewcommand{\section}{\@startsection{section}{1}{\z@}
%% {1.5\Cvs \@plus.5\Cvs \@minus.2\Cvs}
%% {.5\Cvs \@plus.3\Cvs}
%% {\reset@font\Large\mcfamily}}
%% \makeatother

%タイトル

\title{「情報科学における論理\footnote{小野寛晰, 日本評論社, 1994}」問題解答集(途中省略有り)}
\setcounter{footnote}{0}
\author{高田 篤司\thanks{神奈川大学理学部情報科学科 田中研究室} \and 原田 崇司\thanks{神奈川大学大学院理学研究科理学専攻 田中研究室}}
\date{\today}
\西暦

%タイトル作成

\begin{document}

\maketitle

\noindent \textbf{問1.1}の解答
\par
1) 正しくない.
\par
1) が正しくないことを証明する為には,$A \supset B$および$A$がともに充足可能であることを仮定して,$B$が充足可能であることを示せば良い.
\par
よって,初めに,
\begin{equation}
 \textrm{$A \supset B$および$A$がともに充足可能である}
 \label{eq:hypo1_1}
\end{equation}
と仮定する.そして,
\begin{equation}
 \textrm{論理式$A$を$p$,論理式$B$を$r \land \lnot r$}
 \label{eq:hypo1_2}
\end{equation}
と仮定する.
\par
仮定(\ref{eq:hypo1_2})より,論理式$A \supset B$,即ち,$p \supset r \land \lnot r$の真理値表は表\ref{tb:AimpB}となる.
\begin{table}[!htbp]
 \centering
   \caption{$p \supset r \land \lnot r \ \ (A \supset B)$の真理値表}
   \vspace{3mm}
   \begin{tabular}{c|c|c|c|c|c}
    $p$ & $r$ & $\lnot r$ & $p$ & $r \land \lnot r$ & $p \supset r \land \lnot r$ \\ \hline
    $t$ & $t$ & $f$       & $t$       & $f$                     & $f$ \\ \hline
    $t$ & $f$ & $t$       & $t$       & $f$                     & $f$ \\ \hline
    $f$ & $t$ & $f$       & $f$       & $f$                     & $t$ \\ \hline
    $f$ & $f$ & $f$       & $f$       & $f$                     & $t$ 
   \end{tabular}
   \label{tb:AimpB}
\end{table}

表\ref{tb:AimpB}より,$A \supset B$は充足可能である.
\par
さらに,表\ref{tb:AimpB}より,$A$は充足可能である.
\par
しかし,表\ref{tb:AimpB}より,$B$は充足可能でない.
\par
以上より,1)は正しくない.

\vspace{5mm}
2) 正しい.
\par
2) が正しいことを証明する為には,$A \supset B$がトートロジで$A$が充足可能であることを仮定して,$B$が充足可能であることを示せば良い.よって,初めに,
\begin{equation}
 \textrm{$A \supset B$がトートロジで$A$充足可能である}
 \label{eq:hypo1_3}
\end{equation}
と仮定する.
\par
仮定(\ref{eq:hypo1_3})より,$A \supset B$ がトートロジーで$A$が充足可能なので,$v(A) = t, \ v(A \supset B) = t$を満たす付値$v$が存在する.
\par
ここで,表\ref{tb:imply}より,$v(A)=t \ \land \ v(A \supset B) = t$ならば,$v(B) = t$である.
\begin{table}[!htbp]
 \centering
   \caption{$A \supset B$の真理値表}
   \vspace{3mm}
   \begin{tabular}{c|c|c}
    $A$ & $B$ & $A \supset B$ \\ \hline
    $t$ & $t$ & $t$ \\ \hline
    $t$ & $f$ & $f$ \\ \hline
    $f$ & $t$ & $t$ \\ \hline
    $f$ & $f$ & $t$ 
   \end{tabular}
   \label{tb:imply}
\end{table}


よって,$A \supset B$がトートロジーで$A$が充足可能なとき,$v(B) = t$となる付値$v$が存在するので,$B$も充足可能である.
\par
以上より,2)は正しい.

\vspace{5mm}
\par

\noindent \textbf{問1.2}の解答
\par
表\ref{tb:porq}より,$(v(p), v(q), v(r)) = (t,f,f)$若しくは,$(v(p), v(q), v(r)) = (f,t,f)$の組を与えればよい.

\begin{table}[!htbp]
  \centering
  \caption{$((p \lor q) \supset r) \lor (p \land q)$の真理値表}
  \vspace{3mm}
  \begin{tabular}{c|c|c|c|c|c|c}
    $p$ & $q$ & $r$ & $p \lor q$ & $(p \lor q) \supset r$ & $p \land q$ & $((p \lor q) \supset r) \lor (p \land q)$ \\ \hline
    $t$ & $t$ & $t$ & $t$ & $t$ & $t$ & $t$ \\ \hline  %1
    $t$ & $t$ & $f$ & $t$ & $f$ & $t$ & $t$ \\ \hline  %2
    $t$ & $f$ & $t$ & $t$ & $t$ & $f$ & $t$ \\ \hline  %3
    $t$ & $f$ & $f$ & $t$ & $f$ & $f$ & $f$ \\ \hline  %4
    $f$ & $t$ & $t$ & $t$ & $t$ & $f$ & $t$ \\ \hline  %5
    $f$ & $t$ & $f$ & $t$ & $f$ & $f$ & $f$ \\ \hline  %6
    $f$ & $f$ & $t$ & $f$ & $t$ & $f$ & $t$ \\ \hline  %7
    $f$ & $f$ & $f$ & $f$ & $t$ & $f$ & $t$ \\         %8
  \end{tabular}
  \label{tb:porq}
\end{table}

%% \par
%% \vspace{5mm}
\newpage

\noindent \textbf{問1.3}の解答 \par 
\vspace{2mm}
論理結合子として$\supset$と$\land$だけを用いる任意の論理式を$L$と表す.
\par
$L$に現れる全ての論理式に$t$を割り当てるような付値$v$を与えれば,$\supset$の真理値表\ref{tb:imply}と$\land$の真理値表\ref{tb:and}より,$v(L) = t$となる.
\par
よって,$L$を真とするような付値$v$が存在するので,論理結合子として$\supset$と$\land$のみを含むようなすべての論理式は充足可能である.
\begin{table}[!htbp]
  \centering
  \caption{$A \supset B$の真理値表}
  \vspace{3mm}
  \begin{tabular}{c|c|c}
    $A$ & $B$ & $A \land B$ \\ \hline
    $t$ & $t$ & $t$ \\ \hline
    $t$ & $f$ & $f$ \\ \hline
    $f$ & $t$ & $f$ \\ \hline
    $f$ & $f$ & $f$ 
  \end{tabular}
  \label{tb:and}
\end{table}

\par
\vspace{5mm}

\noindent \textbf{問1.4}の解答 \par 
表\ref{tb:equiv}の$A \equiv B$の真理値表より$A$と$B$の真偽が等しいとき,またそのときに限り$A \equiv B$は真となっているので,任意の付値$v$に対して.$v(A \equiv B) = t \  \Longleftrightarrow \ v(A) = v(B)$である.

\begin{table}[!htbp]
  \centering
  \caption{$A \equiv B \ (\Longleftrightarrow (A \supset B) \land (B \supset A))$の真理値表}
  \vspace{3mm}
  \begin{tabular}{c|c|c|c|c}
    $A$ & $B$ & $A \supset B$ & $B \supset A$ & $A \equiv B$ \\ \hline
    $t$ & $t$ & $t$ & $t$ & $t$  \\ \hline
    $t$ & $f$ & $f$ & $t$ & $f$  \\ \hline
    $f$ & $t$ & $t$ & $f$ & $f$  \\ \hline
    $f$ & $f$ & $t$ & $t$ & $t$  \\ 
  \end{tabular}
  \label{tb:equiv}
\end{table}

\par
\vspace{5mm}

\noindent \textbf{問1.5}の解答 \par 
\textbf{問1.4}と同様に,$(A \land B) \lor (\lnot A \land \lnot B)$の真理値表を書くことにより示す.
\par
$(A \land B) \lor (\lnot A \land \lnot B)$の真理値表\ref{tb:equiv2}より,
\begin{equation}
\textrm{任意の付値$v$に対して$A$と$B$の真偽が等しいとき,またそのときに限り$(A \land B) \lor (\lnot A \land \lnot B)$は真となる.}
 \label{eq:151}
\end{equation}
\par
\textbf{問1.4}より,
\begin{equation}
\textrm{任意の付値$v$に対して$A$と$B$の真偽が等しいとき,またそのときに限り$A \equiv B$は,真となる.}
 \label{eq:152}
\end{equation}
\par
(\ref{eq:151}), (\ref{eq:152})より,任意の付値$v$に対して$A \equiv B$が真のとき,またそのときに限り$(A \land B) \lor (\lnot A \land \lnot B)$は真となる.
\par
即ち,任意の付値$v$に対して$v(A \equiv B) = v((A \land B) \lor (\lnot A \land \lnot B))$となる.

\begin{table}[!htbp]
  \centering
  \caption{$(A \land B) \lor (\lnot A \land \lnot B)$の真理値表}
  \vspace{3mm}
  \begin{tabular}{c|c|c|c|c|c|c}
    $A$ & $B$ & $A \land B$ & $\lnot A$ & $\lnot B$ & $\lnot A \land \lnot B$ & $(A \land B) \lor (\lnot A \land \lnot B)$
    \\ \hline
    $t$ & $t$ & $t$ & $f$ & $f$ & $f$ & $t$ \\ \hline
    $t$ & $f$ & $f$ & $f$ & $t$ & $f$ & $f$ \\ \hline
    $f$ & $f$ & $f$ & $t$ & $t$ & $t$ & $f$ \\ \hline
    $f$ & $f$ & $f$ & $t$ & $f$ & $f$ & $t$ \\ 
  \end{tabular}
  \label{tb:equiv2}
\end{table}

\par
\vspace{5mm}

\noindent \textbf{問1.6} 1)の解答 \par 
\begin{enumerate}
 \item 真理値表\ref{tb:id1}より,$A \land A \equiv A$はトートロジーである. \\
  真理値表\ref{tb:id2}より,$A \lor A \equiv A$はトートロジーである. 
\begin{table}[!htbp]
 \centering
 \caption{$(A \supset A \land A) \land (A \land A \supset A)$の真理値表}
 \vspace{3mm}
 \begin{tabular}{c|c|c|c|c}
   $A$ & $A \land A$ & $A \supset A \land A$ & $A \land A \supset A$ & $(A \supset A \land A) \land (A \land A \supset A)$ \\ \hline
   $t$ & $t$ & $t$ & $t$ & $t$ \\ \hline
   $f$ & $f$ & $t$ & $t$ & $t$ 
 \end{tabular}
 \label{tb:id1}
\end{table}

\begin{table}[!htbp]
 \centering
 \caption{$(A \supset A \lor A) \land (A \lor A \supset A)$の真理値表}
 \vspace{3mm}
 \begin{tabular}{c|c|c|c|c}
   $A$ & $A \lor A$ & $A \supset A \lor A$ & $A \lor A \supset A$ & $(A \supset A \lor A) \land (A \lor A \supset A)$ \\ \hline
   $t$ & $t$ & $t$ & $t$ & $t$ \\ \hline
   $f$ & $f$ & $t$ & $t$ & $t$ 
 \end{tabular}
 \label{tb:id2}
\end{table}
 \item 真理値表\ref{tb:associative1}より,任意の付値$v$で$v(\underline{(A \land (B \land C) \supset (A \land B) \land C)} \land \underline{((A \land B) \land C \supset A \land (B \land C))}) = t$なので,$A \land (B \land C) \equiv (A \land B) \land C$はトートロジーである.\\
  また,真理値表\ref{tb:associative2}より,任意の付値$v$で$v(\underline{(A \lor (B \lor C) \supset (A \lor B) \lor C)} \land \underline{((A \lor B) \lor C \supset A \lor (B \lor C))}) = t$なので,$A \lor (B \lor C) \equiv (A \lor B) \lor C$はトートロジーである.\\
\begin{table}[!htbp]
  {\small
    \caption{$A \land (B \land C)$と$(A \land B) \land C$の真理値表}
    \vspace{3mm}
    \begin{tabular}{c|c|c|c|c|c|c|c|c}
      $A$ & $B$ & $C$ & $B \land C$ & $A \land (B \land C)$ & $A \land B$ & $(A \land B) \land C$ & $A \land (B \land C) \supset (A \land B) \land C$ & $(A \land B) \land C \supset A \land (B \land C)$\\ \hline
      $t$ & $t$ & $t$ & $t$ & $t$ & $t$ & $t$ & $t$ & $t$ \\ \hline
      $t$ & $t$ & $f$ & $f$ & $f$ & $t$ & $f$ & $t$ & $t$ \\ \hline
      $t$ & $f$ & $t$ & $f$ & $f$ & $f$ & $f$ & $t$ & $t$ \\ \hline
      $t$ & $f$ & $f$ & $f$ & $f$ & $f$ & $f$ & $t$ & $t$ \\ \hline
      $f$ & $t$ & $t$ & $t$ & $f$ & $f$ & $f$ & $t$ & $t$ \\ \hline
      $f$ & $t$ & $f$ & $f$ & $f$ & $f$ & $f$ & $t$ & $t$ \\ \hline
      $f$ & $f$ & $t$ & $f$ & $f$ & $f$ & $f$ & $t$ & $t$ \\ \hline
      $f$ & $f$ & $f$ & $f$ & $f$ & $f$ & $f$ & $t$ & $t$
    \end{tabular}
    \label{tb:associative1}
  }
\end{table}

\begin{table}[!htbp]
  \centering
      {\small
        \caption{$A \lor (B \lor C)$と$(A \lor B) \lor C$の真理値表}
        \vspace{3mm}
        \begin{tabular}{c|c|c|c|c|c|c|c|c}
          $A$ & $B$ & $C$ & $B \lor C$ & $A \lor (B \lor C)$ & $A \lor B$ & $(A \lor B) \lor C$ & $A \lor (B \lor C) \supset (A \lor B) \lor C$ & $(A \lor B) \lor C \supset A \lor (B \lor C)$\\ \hline
          $t$ & $t$ & $t$ & $t$ & $t$ & $t$ & $t$ & $t$ & $t$ \\ \hline
          $t$ & $t$ & $f$ & $t$ & $t$ & $t$ & $t$ & $t$ & $t$ \\ \hline
          $t$ & $f$ & $t$ & $t$ & $t$ & $t$ & $t$ & $t$ & $t$ \\ \hline
          $t$ & $f$ & $f$ & $f$ & $t$ & $t$ & $t$ & $t$ & $t$ \\ \hline
          $f$ & $t$ & $t$ & $t$ & $t$ & $t$ & $t$ & $t$ & $t$ \\ \hline
          $f$ & $t$ & $f$ & $t$ & $t$ & $t$ & $t$ & $t$ & $t$ \\ \hline
          $f$ & $f$ & $t$ & $t$ & $t$ & $f$ & $t$ & $t$ & $t$ \\ \hline
          $f$ & $f$ & $f$ & $f$ & $f$ & $f$ & $f$ & $t$ & $t$
        \end{tabular}
        \label{tb:associative2}
      }
\end{table}
 \item 真理値表\ref{tb:commutative1}より,$A \land B \equiv B \land A$はトートロジーである.\\ 
  真理値表\ref{tb:commutative2}より,$A \lor B \equiv B \lor A$はトートロジーである.
\begin{table}[!htbp]
  \centering
  \caption{$A \land B \equiv B \land A$の真理値表}
  \vspace{3mm}
  \begin{tabular}{c|c|c|c|c|c|c}
    $A$ & $B$ & $A \land B$ & $B \land A$ & $A \land B \supset B \land A$ & $B \land A \supset A \land B$ & $(A \land B \supset B \land A) \ \land \ (B \land A \supset A \land B)$ \\ \hline
    $t$ & $t$ & $t$ & $t$ & $t$ & $t$ & $t$ \\ \hline 
    $t$ & $f$ & $f$ & $f$ & $t$ & $t$ & $t$ \\ \hline
    $f$ & $t$ & $f$ & $f$ & $t$ & $t$ & $t$ \\ \hline 
    $f$ & $f$ & $f$ & $f$ & $t$ & $t$ & $t$
  \end{tabular}
  \label{tb:commutative1}
\end{table}

\begin{table}[!htbp]
  \centering
  \caption{$A \lor B \equiv B \lor A$の真理値表}
  \vspace{3mm}
  \begin{tabular}{c|c|c|c|c|c|c}
    $A$ & $B$ & $A \lor B$ & $B \lor A$ & $A \lor B \supset B \lor A$ & $B \lor A \supset A \lor B$ & $(A \lor B \supset B \lor A) \ \land \ (B \lor A \supset A \lor B)$ \\ \hline
    $t$ & $t$ & $t$ & $t$ & $t$ & $t$ & $t$ \\ \hline 
    $t$ & $f$ & $t$ & $t$ & $t$ & $t$ & $t$ \\ \hline
    $f$ & $t$ & $t$ & $t$ & $t$ & $t$ & $t$ \\ \hline 
    $f$ & $f$ & $f$ & $f$ & $t$ & $t$ & $t$
  \end{tabular}
  \label{tb:commutative2}
\end{table}

%4 吸収律
 \item 真理値表\ref{tb:absorption1}より,$A \land (A \lor B) \equiv A$はトートロジーである.\\ 
  真理値表\ref{tb:absorption2}より,$A \lor (A \land B) \equiv A$はトートロジーである.
\begin{table}[!htbp]
 \centering
 \caption{$A \land (A \lor B) \equiv A$の真理値表}
 \vspace{3mm}
 \begin{tabular}{c|c|c|c|c|c|c}
   $A$ & $B$ & $A \lor B$ & $A \land (A \lor B)$ & $A \land (A \lor B) \supset A$ & $A \supset A \land (A \lor B)$ & $(A \land (A \lor B) \supset A) \land (A \supset A \land (A \lor B)) $ \\ \hline 
   $t$ & $t$ & $t$ & $t$ & $t$ & $t$ & $t$ \\ \hline 
   $t$ & $f$ & $t$ & $t$ & $t$ & $t$ & $t$ \\ \hline
   $f$ & $t$ & $t$ & $f$ & $t$ & $t$ & $t$ \\ \hline 
   $f$ & $f$ & $f$ & $f$ & $t$ & $t$ & $t$
 \end{tabular}
 \label{tb:absorption1}
\end{table}

\begin{table}[!htbp]
  \centering
  \caption{$A \lor (A \land B) \equiv A$の真理値表}
  \vspace{3mm}
  \begin{tabular}{c|c|c|c|c|c|c}
    $A$ & $B$ & $A \land B$ & $A \lor (A \land B)$ & $A \lor (A \land B) \supset A$ & $A \supset A \lor (A \land B)$ & $(A \land (A \lor B) \supset A) \land (A \supset A \lor (A \land B)) $ \\ \hline 
    $t$ & $t$ & $t$ & $t$ & $t$ & $t$ & $t$ \\ \hline 
    $t$ & $f$ & $f$ & $t$ & $t$ & $t$ & $t$ \\ \hline
    $f$ & $t$ & $f$ & $f$ & $t$ & $t$ & $t$ \\ \hline 
    $f$ & $f$ & $f$ & $f$ & $t$ & $t$ & $t$
  \end{tabular}
  \label{tb:absorption2}
\end{table}
 \item
 \item
 \item
 \item
\end{enumerate}

\par
\vspace{5mm}

\noindent \textbf{問1.7}の解答
\par
\begin{enumerate}
\item $C$が$D \lor E$の形のとき,\par $C_{A}$および$C_{B}$はそれぞれ$D_{A} \lor E_{A}$および$D_{B} \lor E_{B}$である.$D$と$E$はともに$C$よりも「簡単」な論理式だから,仮定より$D_{A} \sim D_{B}, E_{A} \sim E_{B}$がともになりたつ.つまり,どんな付値$v$をとっても$v(D_{A}) = v(D_{B})$および$v(E_{A}) = v(E_{B})$となる.ところが
  \begin{align*}
    v(D_{A} \lor E_{A}) = t & \iff v(D_{A}) = t \ {\rm または} \ v(E_{A}) = t\\
    & \iff v(D_{B}) = t \ {\rm または} \ v(E_{B}) = t\\
    & \iff v(D_{B} \lor E_{B}) = t
  \end{align*}
  より,$v(C_{A}) = v(C_{B})$となる.$v$は任意の付値だから,$C_{A} \sim C_{B}$が得られる.
\item $C$が$D \supset  E$の形のとき,\par $C_{A}$および$C_{B}$はそれぞれ$D_{A} \supset E_{A}$および$D_{B} \supset E_{B}$である.$D$と$E$はともに$C$よりも「簡単」な論理式だから,仮定より$D_{A} \sim D_{B}, E_{A} \sim E_{B}$がともになりたつ.つまり,どんな付値$v$をとっても$v(D_{A}) = v(D_{B})$および$v(E_{A}) = v(E_{B})$となる.ところが
  \begin{align*}
    v(D_{A} \supset E_{A}) = t & \iff v(D_{A}) = f \ {\rm または} \  v(E_{A}) = t\\
    & \iff v(D_{B}) = f \ {\rm または} \  v(E_{B}) = t\\
    & \iff v(D_{B} \supset E_{B}) = t
  \end{align*}
  より,$v(C_{A}) = v(C_{B})$となる.$v$は任意の付値だから,$C_{A} \sim C_{B}$が得られる.
\item $C$が$\lnot D$の形のとき,\par $C_{A}$は$\lnot D_{A}$である.$D$は$C$よりも「簡単」な論理式だから,仮定より$D_{A} \sim D_{B}$がともになりたつ.つまり,どんな付値$v$をとっても$v(D_{A}) = v(D_{B})$となる.ところが
  \begin{align*}
    v(\lnot D_{A}) = t & \iff v(D_{A}) = f \\
    & \iff v(D_{B}) = f \\
    & \iff v(\lnot D_{B}) = t
  \end{align*}
  より,$v(C_{A}) = v(C_{B})$となる.$v$は任意の付値だから,$C_{A} \sim C_{B}$が得られる.
\end{enumerate}

\par
\vspace{5mm}

\noindent \textbf{問1.8}の解答
\par
$p \supset (q \supset r) \sim (p \land q) \supset r$がなりたつことを,論理式$p \supset (q \supset r)$を同値変形で置き換えて論理式$(p \land q) \supset r$へと置き換えることにより示す.
\begin{align*}
p \supset (q \supset r) \ &\sim \ \lnot p \lor (q \supset r) \ \ {\rm (定理1.3.8より)} \\
                        \ &\sim \ \lnot p \lor (\lnot q \lor r) \ \ {\rm (定理1.3.8より)} \\
                        \ &\sim \ (\lnot p \lor \lnot q) \lor r \ \ {\rm (}\lor {\rm の結合法則より)} \\
                        \ &\sim \ (\lnot(p \land q)) \lor r \ \ {\rm (De \ Morganの法則より)} \\
                        \ &\sim \ (p \land q) \supset r \ \ {\rm (De \ Morganの法則より)} \\
\end{align*}
以上の置き換えより,$p \supset (q \supset r) \sim (p \land q) \supset r$がなりたつ.

\par
\vspace{5mm}

\noindent \textbf{問1.9}の解答
\par
\begin{enumerate}
\item
$\lnot(p \supset (q \land r))$と$(p \land q \land \lnot r) \lor (p \land \lnot q \land r) \lor (p \land \lnot q \land \lnot r)$の真理値表を書いて,それぞれを照らし合わせて$\lnot(p \supset (q \land r)) \ \sim \ (p \land q \land \lnot r) \lor (p \land \lnot q \land r) \lor (p \land \lnot q \land \lnot r)$となることを確かめる.
\begin{table}[!htbp]
\centering
\caption{$\lnot(p \supset (q \land r))$の真理値表}
\vspace{3mm}
\begin{tabular}{c|c|c|c|c|c}
  $p$ & $q$ & $r$ & $q \land r$ & $p \supset (q \land r)$ & $\lnot (p \supset (q \land r))$ \\ \hline
  $t$ & $t$ & $t$ & $t$         & $t$                     & $f$ \\ \hline
  $t$ & $t$ & $f$ & $f$         & $f$                     & $t$ \\ \hline
  $t$ & $f$ & $t$ & $f$         & $f$                     & $t$ \\ \hline
  $t$ & $f$ & $f$ & $f$         & $f$                     & $t$ \\ \hline
  $f$ & $t$ & $t$ & $t$         & $t$                     & $f$ \\ \hline
  $f$ & $t$ & $f$ & $f$         & $t$                     & $f$ \\ \hline
  $f$ & $f$ & $t$ & $f$         & $t$                     & $f$ \\ \hline
  $f$ & $f$ & $f$ & $f$         & $t$                     & $f$ 
\end{tabular}
\label{tb:1_9_1}
\end{table}

\begin{table}[!htbp]
\centering
  \caption{$(p \land q \land \lnot r) \lor (p \land \lnot q \land r) \lor (p \land \lnot q \land \lnot r)$の真理値表}
  \vspace{3mm}
  \begin{tabular}{c|c|c|c|c|c|c}
   $p$ & $q$ & $r$ & $p \land q \land \lnot r$ & $p \land \lnot q \land r$ & $p \land \lnot q \land \lnot r$ & $(p \land q \land \lnot r) \lor (p \land \lnot q \land r) \lor (p \land \lnot q \land \lnot r)$ \\ \hline
   $t$ & $t$ & $t$ & $f$ & $f$ & $f$ & $f$ \\ \hline
   $t$ & $t$ & $f$ & $t$ & $f$ & $f$ & $t$ \\ \hline
   $t$ & $f$ & $t$ & $f$ & $t$ & $f$ & $t$ \\ \hline 
   $t$ & $f$ & $f$ & $f$ & $f$ & $t$ & $t$ \\ \hline
   $f$ & $t$ & $t$ & $f$ & $f$ & $f$ & $f$ \\ \hline 
   $f$ & $t$ & $f$ & $f$ & $f$ & $f$ & $f$ \\ \hline
   $f$ & $f$ & $t$ & $f$ & $f$ & $f$ & $f$ \\ \hline
   $f$ & $f$ & $f$ & $f$ & $f$ & $f$ & $f$
  \end{tabular}
  \label{tb:1_9_2}
\end{table}

$\lnot(p \supset (q \land r))$の真理値表と$(p \land q \land \lnot r) \lor (p \land \lnot q \land r) \lor (p \land \lnot q \land \lnot r)$の真理値表を見ると,どちらの論理式も$(p = t, q = t, r = f), (p = t, q = f, r = t), (p = t, q = f, r = f)$の割り当てのときのみ$t$となる.つまり,任意の付置$v$に対して二つの論理式の真偽が一致しているので,$\lnot(p \supset (q \land r)) \ \sim \ (p \land q \land \lnot r) \lor (p \land \lnot q \land r) \lor (p \land \lnot q \land \lnot r)$となる.
\item \today の時点において分かっていない.
\end{enumerate}

\vspace{5mm}
\par

\noindent \textbf{問1.10}の解答
\par
それぞれの論理式を例1.10のように論理和標準形および論理積標準形置き換える.
\begin{enumerate}
 \item $((p \supset q) \supset p)) \supset p$
  \begin{align*}
   ((p \supset q) \supset p)) \supset p \ &\sim \ ((\lnot p \lor q) \supset p)) \supset p \hspace{10mm} {\rm (定理1.3.8より)} \\
    &\sim \ (\lnot (\lnot p \lor q) \lor p)) \supset p \hspace{10mm} {\rm (定理1.3.8より)} \\
    &\sim \ \lnot (\lnot (\lnot p \lor q) \lor p) \lor p \hspace{10mm} {\rm (定理1.3.8より)} \\
    &\sim \ \lnot ((\lnot \lnot p \land \lnot q) \lor p) \lor p \hspace{10mm} {\rm (de \ Morganの法則より)} \\
    &\sim \ (\lnot(\lnot \lnot p \land \lnot q) \land \lnot p) \lor p \hspace{10mm} {\rm (de \ Morganの法則より)} \\
    &\sim \ ((\lnot (\lnot \lnot p) \lor \lnot \lnot q) \land \lnot p) \lor p \hspace{10mm} {\rm (de \ Morganの法則より)} \\
    &\sim \ ((\lnot p \lor q) \land \lnot p) \lor p \hspace{10mm} {\rm (定理1.3.6より)} \\
    &\sim \ ((\lnot p \land \lnot p) \lor (q \land \lnot p)) \lor p \hspace{10mm} {\rm (分配律より)} \\
    &\sim \ (\lnot p \lor (q \land \lnot p)) \lor p \hspace{10mm} {\rm (冪等律より)\hspace{30mm} 「論理和標準形」} \\
    &\sim \ ((\lnot p \lor q) \land (\lnot p \lor \lnot p)) \lor p \hspace{10mm} {\rm (分配律より)} \\
    &\sim \ ((\lnot p \lor q) \land \lnot p) \lor p \hspace{10mm} {\rm (冪等律より)} \\
    &\sim \ ((\lnot p \lor q) \lor p) \land (\lnot p \lor p) \hspace{10mm} {\rm (分配律より)} \\
    &\sim \ ((\lnot p \lor p) \lor q) \land (\lnot p \lor p) \hspace{10mm} {\rm (結合律と交換律より)} \\ 
    &\sim \ (\top \lor q) \land \top  \hspace{10mm} {\rm (定理1.3.9より)\hspace{37mm} 「論理積標準形?」}
%    &\sim \ (\top \lor q) \ \ {\rm (定理1.3.11より)} \\
%    &\sim \ \top \ \ {\rm (定理1.3.10より) \hspace{50mm} 「論理積標準形?」} \\
  \end{align*}
 「論理和標準形」と注を付けている行の3行下の行は,節に同一の論理変数が二つ以上現れているので「論理積標準形」と見做さない.
 \item $(p \supset (p \land \lnot q)) \land (q \supset (q \land \lnot p)))$
  \begin{align*}
   & (p \supset (p \land \lnot q)) \land (q \supset (q \land \lnot p))) \ \sim \ (\lnot p \lor (p \land \lnot q)) \land (q \supset (q \land \lnot p))) \hspace{10mm} {\rm (定理1.3.8より)}  \\
   & \sim \ (\lnot p \lor (p \land \lnot q)) \land (\lnot q \lor (q \land \lnot p))) \hspace{10mm} {\rm (定理1.3.8より)}  \\
   & \sim \ ((\lnot p \lor (p \land \lnot q)) \land \lnot q) 
      \lor ((\lnot p \lor (p \land \lnot q)) \land (q \land \lnot p))) \hspace{10mm} {\rm (分配律より)}  \\
   & \sim \ ((\lnot p \land \lnot q) \lor ((p \land \lnot q) \land \lnot q)) 
      \lor ((\lnot p \lor (p \land \lnot q)) \land (q \land \lnot p))) \hspace{10mm} {\rm (分配律より)}  \\
   & \sim \ ((\lnot p \land \lnot q) \lor (p \land \lnot q)) 
      \lor ((\lnot p \lor (p \land \lnot q)) \land (q \land \lnot p))) \hspace{10mm} {\rm (冪等律より)}  \\
   & \sim \ ((\lnot p \land \lnot q) \lor (p \land \lnot q)) \\
   & \ \ \ \lor ((\lnot p \land (q \land \lnot p)) \lor ((p \land \lnot q) \land (q \land \lnot p))) \hspace{10mm} {\rm (分配律より)}  \\
   & \sim \ ((\lnot p \land \lnot q) \lor (p \land \lnot q)) \\
   & \ \ \ \lor ((\lnot p \land q) \lor ((p \land \lnot p) \land (q \land \lnot q))) \hspace{10mm} {\rm (交換律と結合律と冪等律より)}  \\
   & \sim \ ((\lnot p \land \lnot q) \lor (p \land \lnot q)) \lor ((\lnot p \land q) \lor (\bot \land \bot)) \hspace{10mm} {\rm (定理1.3.9より)}  \\
   & \sim \ ((\lnot p \land \lnot q) \lor (p \land \lnot q)) \lor (\lnot p \land q) \hspace{10mm} {\rm (定理1.3.11より)\hspace{10mm} 「論理和標準形」}  \\
   & \sim \ ((\lnot p \lor (p \land \lnot q) \lor (\lnot q \lor (p \land \lnot q)) \lor (\lnot p \land q) \hspace{10mm} {\rm (分配律より)} \\
   & \sim \ ((\lnot p \lor p) \land (\lnot p \lor \lnot q) \lor (\lnot q \lor (p \land \lnot q)) \lor (\lnot p \land q) \hspace{10mm} {\rm (分配律より)} \\
   & \sim \ (\top \land (\lnot p \lor \lnot q) \lor (\lnot q \lor (p \land \lnot q)) \lor (\lnot p \land q) \hspace{10mm} {\rm (定理1.3.9より)} \\
   & \sim \ ((\lnot p \lor \lnot q) \lor (\lnot q \lor (p \land \lnot q)) \lor (\lnot p \land q) \hspace{10mm} {\rm (定理1.3.9より)} \\
   & \sim \ ((\lnot p \lor \lnot q) \lor ((\lnot q \lor p) \land (\lnot q \lor \lnot q))) \lor (\lnot p \land q) \hspace{10mm} {\rm (分配律より)} \\
   & \sim \ ((\lnot p \lor \lnot q) \lor ((\lnot q \lor p) \land \lnot q)) \lor (\lnot p \land q) \hspace{10mm} {\rm (冪等律より)} \\
   & \sim \ ((\lnot p \lor \lnot q) \lor (\lnot q \lor p)) \land ((\lnot p \lor \lnot q) \lor \lnot q)) \lor (\lnot p \land q) \hspace{10mm} {\rm (分配律より)} \\
   & \sim \ ((\lnot p \lor \lnot q) \lor p) \land (\lnot p \lor \lnot q)) \lor (\lnot p \land q) \hspace{10mm} {\rm (結合律と冪等律より)} \\
   & \sim \ (\top \land (\lnot p \lor \lnot q)) \lor (\lnot p \land q) \hspace{10mm} {\rm (結合律と交換律と定理1.3.10より)} \\
   & \sim \ (\lnot p \lor \lnot q) \lor (\lnot p \land q) \hspace{10mm} {\rm (結合律と交換律と定理1.3.11より)} \\
   & \sim \ ((\lnot p \lor \lnot q) \lor \lnot p) \land ((\lnot p \lor \lnot q) \lor q) \hspace{10mm} {\rm (分配律より)} \\
   & \sim \ (\lnot p \lor \lnot q) \land (\lnot p \lor (\lnot q \lor q)) \hspace{10mm} {\rm (結合律と交換律と冪等律より)} \\
   & \sim \ (\lnot p \lor \lnot q) \hspace{10mm} {\rm (定理1.3.9, 1.3.10, 1.3.11より)\hspace{10mm} 「論理積標準形」} \\
  \end{align*}
 \item $\lnot (p \supset q) \land ((q \supset s) \supset r)$ \par (省略)
\end{enumerate}

\par
\vspace{5mm}

\noindent \textbf{問1.11}の解答
\par
$A \land B$と$C \lor D$の形の論理式を$\supset$と$\lnot$のみを用いた同値な論理式へ置き換えることによって示す.
\begin{align*}
   A \land B \ &\sim \ \lnot \lnot A \land \lnot \lnot B \hspace{10mm} {\rm (定理1.3.6より)} \\
    &\sim \ \lnot (\lnot A \lor \lnot B) \hspace{10mm} {\rm (De \ Morganの法則より)} \\
    &\sim \ \lnot (A \supset \lnot B) \hspace{10mm} {\rm (定理1.3.8より)} \\
\end{align*}
\begin{align*}
   A \lor B \ &\sim \ \lnot \lnot A \lor B \hspace{10mm} {\rm (定理1.3.6より)} \\
    &\sim \ \lnot A \supset B \hspace{10mm} {\rm (定理1.3.8より)} \\
\end{align*}

\vspace{5mm}
\par

\noindent \textbf{問1.12}の解答
\par
\today の時点において分かっていない.

\vspace{5mm}
\par

\noindent \textbf{問1.13}の解答
\begin{displaymath}
\setcounter{prfassumptioncounter}{0}
{\prftree[r]{(contraction右)}
  {\prftree[r]{($\lor$右1)}
    {\prftree[r]{(exchange右)}
      {\prftree[r]{($\lor$右2)}
        {\prftree[r]{($\lnot$右)}
          {A \to A}
          {\to A, \neg A}
        }
        {\to A, A \lor \neg A}
      }
      {\to A \lor \neg A, A}
    }
    {\to A \lor \neg A, A \lor \neg A}
  }
  {\to A \lor \neg A}
}
\end{displaymath}\vspace{.2ex}

\vspace{5mm}
\par

\noindent \textbf{問1.14}の解答
\begin{displaymath}
\setcounter{prfassumptioncounter}{0}
{\prftree[r]{($\neg$右)}
  {\prftree[r]{(cont.左)}
    {\prftree[r]{($\land$左1)}
      {\prftree[r]{(ex.左)}
        {\prftree[r]{($\land$左2)}
          {\prftree[r]{(ex.左)}
            {\prftree[r]{($\neg$左)}
              {\prftree[r]{($\supset$左)}
                {A \to A}
                {B \to B}
                {A \supset B, A \to B}
              }
              {\neg B, A \supset B, A \to}
            }
            {\neg B, A, A \supset B \to}
          }
          {A \land \neg B, A, A \supset B \to}
        }
        {A, A \land \neg B, A \supset B \to}
      }
      {A \land \neg B, A \land \neg B, A \supset B \to}
    }
    {A \land \neg B, A \supset B \to}
  }
  {A \supset B \to \neg (A \land \neg B)}
}
\end{displaymath}\vspace{.2ex}

\vspace{5mm}
\par

\noindent \textbf{問1.15}の解答
\begin{displaymath}
\setcounter{prfassumptioncounter}{0}
{\prftree[r]{(cont.右)}
  {\prftree[r]{($\lor$右1)}
    {\prftree[r]{($\neg$右)}
      {\prftree[r]{($\lor$右2)}
        {A, \Gamma \to \Delta, B}
        {A, \Gamma \to \Delta, \neg A \lor B}
      }
      {\Gamma \to \Delta, \neg A \lor B, \neg A}
    }
    {\Gamma \to \Delta, \neg A \lor B, \neg A \lor B}
  }
  {\Gamma \to \Delta, \neg A \lor B}
}
\end{displaymath}\vspace{.2ex}

\vspace{5mm}
\par

\noindent \textbf{問1.16}の解答
\begin{enumerate}
  \renewcommand{\labelenumi}{\arabic{enumi}) }
\item
\begin{displaymath}
\setcounter{prfassumptioncounter}{0}
{\prftree[r]{($\neg$左)}
  {\prftree[r]{($\land$右)}
    {\prftree[r]{(ex.右)}
      {\prftree[r]{($\lor$右1)}
        {\prftree[r]{($\neg$右)}
          {A \to A}
          {\to A, \neg A}
        }
        {\to A, \neg A \lor \neg B}
      }
      {\to \neg A \lor \neg B, A}
    }
    {\prftree[r]{(ex.右)}
      {\prftree[r]{($\lor$右2)}
        {\prftree[r]{($\neg$右)}
          {B \to B}
          {\to B, \neg B}
        }
        {\to B, \neg A \lor \neg B}
      }
      {\to \neg A \lor \neg B, B}
    }
    {\to \neg A \lor \neg B, A \land B}
  }
  {\neg (A \land B) \to \neg A \lor \neg B}
}
\end{displaymath}\vspace{.2ex}
\item
\item
\item
\end{enumerate}

\vspace{5mm}
\par

\newpage

\renewcommand{\labelenumi}{(\arabic{enumi}) }
\renewcommand{\labelenumii}{\arabic{enumii}) }

\noindent \textbf{問2.1}の解答 
\begin{enumerate}
 \item
  \begin{enumerate}
   \item すべての教師を好きな学生がいる
   \item すべての怠け者の学生は,怠け者の教師が好きではない
  \end{enumerate}
 \item $\neg (\forall x \ (T(x) \supset L(x)))$
\end{enumerate}

\par
\vspace{5mm}

\noindent \textbf{問2.2}の解答 
\begin{enumerate}
 \item 実数の集合は稠密順序集合である(任意の実数のいくらでも近くに別の実数が存在する).
 \item 関係$<$は推移律を満たす.
\end{enumerate}

\par
\vspace{5mm}

\noindent \textbf{問2.5}の解答 
\renewcommand{\labelenumi}{(\arabic{enumi}) }
\renewcommand{\labelenumii}{\arabic{enumii}) }
\begin{enumerate}
 \item 
  \begin{enumerate}
   \item 成り立たない.何故ならば$a_{4}Ry$,$a_{5}Ry$となるような$y$は存在しないから.
   \item 成り立つ.任意の$x$に対して,$y$として$a_{4}$か$a_{5}$をとる($x := a_{4}$なら$y := a_{4}$,$x := a_{5}$なら$y := a_{5}$とする).そうすると$yRz$を満たすような$z$は適当に選んだ$y$自身しかないので$yRy \supset yRy$となる.$R$は順序関係なので反射律を満たすのでこの論理式は成り立つ.
  \end{enumerate}
 \item 
 \begin{enumerate}
   \setcounter{enumii}{2}
   \item 成り立つ.$y := a_{4}, z := a_{5}$とすれば良い.
   \item 成り立たない.$x := a_{1}, y := a_{2}, z := a_{3}$とすると,$xRy \land yRz$は成り立つが,$x = y \lor y = z$は成り立たない.つまり,$\forall x \, \forall y \, \forall z \, ((xRy \land yRz) \supset (x=y \lor y = z))$は成り立たない.
 \end{enumerate}
\end{enumerate}

\par
\vspace{5mm}

\noindent \textbf{問2.8}の解答 
\begin{enumerate}
 \item 
  \begin{enumerate}
   \item $\models \forall x \, A \equiv A$を示す.
    \par \noindent 
    恒真性の定義(p.70)より,
    \par \noindent 
    任意の構造$\mathfrak{A}$に対し,$\mathfrak{A} \models \forall y_{1} \cdots \forall y_{n} \ (\forall x \, A \equiv A)$となることをいえばよい.ここで論理式$\forall x \ A \equiv A$は自由変数として$y_{1} \cdots y_{n}$を持つものとしておく.
さらに,$\equiv$の定義(p.10)より
    \begin{equation*}
     \mathfrak{A} \models \forall y_{1} \cdots \forall y_{n} \ ((\forall x \, A \supset A) \land (A \supset \forall x \, A))
    \end{equation*}
    をいえばよい.これを示すには,任意の構造$\mathfrak{A} = \langle U, I \rangle$および任意の$u_{1}, \dots, u_{n} \in U$に対し
    \begin{align*}
     \mathfrak{A} \models ( & (\forall x \ A[ \underline{u_{1}} / y_{1}, \cdots , \underline{u_{n}} / y_{n}] \supset A[\underline{u_{1}}/y_{1}, \cdots , \underline{u_{n}}/y_{n}]) \\
     & \land \ (A[\underline{u_{1}}/y_{1}, \cdots , \underline{u_{n}}/y_{n}] \supset \forall x \ A[ \underline{u_{1}} / y_{1}, \cdots , \underline{u_{n}} / y_{n}]))
    \end{align*}
をいえばよい.
    \par \noindent
    ここで,$A \land B$の形の論理式が正しいということの意味(p.64(4))より
    \begin{equation*}
     \mathfrak{A} \models (\forall x \ A[ \underline{u_{1}} / y_{1}, \cdots , \underline{u_{n}} / y_{n}] \supset A[\underline{u_{1}}/y_{1}, \cdots , \underline{u_{n}}/y_{n}])
    \end{equation*}
    と
    \begin{equation*}
     \mathfrak{A} \models A[\underline{u_{1}}/y_{1}, \cdots , \underline{u_{n}}/y_{n}] \supset \forall x \ A[ \underline{u_{1}} / y_{1}, \cdots , \underline{u_{n}} / y_{n}]
    \end{equation*}
    が正しいことを示せばよい.以下の証明では,自由変数$x_{1}, \cdots, x_{n}$への$\underline{u_{1}}, \cdots \underline{u_{n}}$への代入がすでに行われているものとし,したがって,$A$が自由変数を一つも含まない場合について述べる.
    \begin{enumerate}
    \renewcommand{\labelenumiii}{\alph{enumiii}) }
     \item 
      $\mathfrak{A} \models \forall x \, A \supset A$を示す.
      \par \noindent
      これを示すには$\mathfrak{A} \models \forall x \, A$を仮定して$\mathfrak{A} \models A$を示せば十分である.
      \par \noindent
      構造$\mathfrak{A} = \langle U, I \rangle$に対して,$\mathfrak{A} \models \forall x \, A$と仮定する.したがって,すべての$u \in U$に対して$\mathfrak{A} \models A[\underline{u}/x]$となる.ここで,$A$は$x$を自由変数として含まないので,すべての$u \in U$に対して,$A$は$A[\underline{u}/x]$に等しい.よって$\mathfrak{A} \models A$である.
     \item 
      $\mathfrak{A} \models A \supset \forall x \, A$を示す.
      \par \noindent
      これを示すには$\mathfrak{A} \models A$を仮定して$\mathfrak{A} \models \forall x \, A$を示せば十分である.
      \par \noindent
      構造$\mathfrak{A} = \langle U, I \rangle$に対して,$\mathfrak{A} \models A$と仮定する.したがって,すべての$u \in U$に対して$\mathfrak{A} \models A$となる.ここで,$A$は$x$を自由変数として含まないので,すべての$u \in U$に対して,$A$は$A[\underline{u}/x]$に等しい.これは,$\mathfrak{A} \models \forall x \, A$である.
    \end{enumerate}
   a), b)より$\models \forall x \, A \equiv A$は恒真である.
   \item $\models \exists x \, A \equiv A$を示す.
    \par \noindent 
    恒真性の定義(p.70)より,
    \par \noindent
    任意の構造$\mathfrak{A}$に対し,$\mathfrak{A} \models \forall y_{1} \cdots \forall y_{n} \ (\exists x \, A \equiv A)$となることをいえばよい.ここで論理式$\exists x \ A \equiv A$は自由変数として$y_{1} \cdots y_{n}$を持つものとしておく.
さらに,$\equiv$の定義(p.10)より
    \begin{equation*}
     \mathfrak{A} \models \forall y_{1} \cdots \forall y_{n} \ ((\exists x \, A \supset A) \land (A \supset \exists x \, A))
    \end{equation*}
    をいえばよい.これを示すには,任意の構造$\mathfrak{A} = \langle U, I \rangle$および任意の$u_{1}, \dots, u_{n} \in U$に対し
    \begin{align*}
     \mathfrak{A} \models ( & (\exists x \ A[ \underline{u_{1}} / y_{1}, \cdots , \underline{u_{n}} / y_{n}] \supset A[\underline{u_{1}}/y_{1}, \cdots , \underline{u_{n}}/y_{n}]) \\
     & \land \ (A[\underline{u_{1}}/y_{1}, \cdots , \underline{u_{n}}/y_{n}] \supset \exists x \ A[ \underline{u_{1}} / y_{1}, \cdots , \underline{u_{n}} / y_{n}]))
    \end{align*}
をいえばよい.
    ここで,$A \land B$の形の論理式が正しいということの意味(p.64(4))より
    \begin{equation*}
     \mathfrak{A} \models (\exists x \ A[ \underline{u_{1}} / y_{1}, \cdots , \underline{u_{n}} / y_{n}] \supset A[\underline{u_{1}}/y_{1}, \cdots , \underline{u_{n}}/y_{n}])
    \end{equation*}
    と
    \begin{equation*}
     \mathfrak{A} \models A[\underline{u_{1}}/y_{1}, \cdots , \underline{u_{n}}/y_{n}] \supset \exists x \ A[ \underline{u_{1}} / y_{1}, \cdots , \underline{u_{n}} / y_{n}]
    \end{equation*}
    が正しいことを示せばよい.以下の証明では,自由変数$x_{1}, \cdots, x_{n}$への$\underline{u_{1}}, \cdots \underline{u_{n}}$への代入がすでに行われているものとし,したがって,$A$が自由変数を一つも含まない場合について述べる.
    \begin{enumerate}
    \renewcommand{\labelenumiii}{\alph{enumiii}) }
     \item 
      $\mathfrak{A} \models \exists x \, A \supset A$を示す.
      \par \noindent
      これを示すためには$\mathfrak{A} \models \exists x \, A$を仮定して$\mathfrak{A} \models A$を示せば十分である.
      \par \noindent
      構造$\mathfrak{A} = \langle U, I \rangle$に対して,$\mathfrak{A} \models \exists x \, A$と仮定する.したがって,ある$u \in U$に対して$\mathfrak{A} \models A[\underline{u}/x]$となる.ここで,$A$は$x$を自由変数として含まないので,$A$は$A[\underline{u}/x]$に等しい.よって$\mathfrak{A} \models A$である.
     \item 
      $\mathfrak{A} \models A \supset \exists x \, A$を示す.
      \par \noindent
      これを示すには$\mathfrak{A} \models A$を仮定して$\mathfrak{A} \models \exists x \, A$を示せば十分である.
      \par \noindent
      構造$\mathfrak{A} = \langle U, I \rangle$に対して,$\mathfrak{A} \models A$と仮定する.したがって,ある$u \in U$に対して$\mathfrak{A} \models A$となる.ここで,$A$は$x$を自由変数として含まないので,ある$u \in U$に対して,$A$は$A[\underline{u}/x]$に等しい.これは,$\mathfrak{A} \models \exists x \, A[x]$である.
    \end{enumerate}
   a), b)より$\models \exists x \, A \equiv A$は恒真である.
  \end{enumerate}
 \item 以下省略
\end{enumerate}

\par
\vspace{5mm}

\noindent \textbf{問2.11}の解答 
\begin{enumerate}
\renewcommand{\labelenumi}{\arabic{enumi}) }
 \item 
  \begin{equation*}
  \begin{array}{lll}
   &\exists x \, R(x,y) \supset \forall y \, (P(y) \land \lnot \, \forall z \, Q(z)) & \\
   \sim \ & \exists x \, R(x,y) \supset \forall u \, (P(u) \land \lnot \, \forall z \, Q(z)) & \hspace{10mm} \mathrm{定理2.1の2)} \\
   \sim \ & \exists x \, \forall u \, (R(x,y) \supset (P(u) \land \exists z \, \lnot \, Q(z))) & \hspace{10mm} \mathrm{定理2.1の11)の左} \\
   \sim \ & \exists x \, \forall u \, (R(x,y) \supset \exists z \, (P(u) \land \lnot \, Q(z))) & \hspace{10mm} \mathrm{定理2.1の4)の右} \\
   \sim \ & \exists x \, \forall u \, \exists z \, (R(x,y) \supset (P(u) \land \lnot \, Q(z))) & \hspace{10mm} \mathrm{定理2.1の11)の右} 
  \end{array}
 \end{equation*}
 \item
 \begin{equation*}
  \begin{array}{lll}
   & \exists x \, (\forall y \, (P(y) \supset Q(x,z)) \lor \exists z \, (\lnot \, (\exists u \, R(z,u) \land Q(x,z)))) \\
\sim & \exists x \, (\forall y \, (P(y) \supset Q(x,z)) \lor \exists v \, (\lnot \, (\exists u \, R(v,u) \land Q(x,v)))) & \hspace{10mm} \mathrm{定理2.1の2)の右} \\
\sim & \exists x \, \exists v \, (\forall y \, (P(y) \supset Q(x,z)) \lor (\lnot \, (\exists u \, R(v,u) \land Q(x,v)))) & \hspace{10mm} \mathrm{定理2.1の3)の右} \\
\sim & \exists x \, \exists v \, (\forall y \, (P(y) \supset Q(x,z)) \lor (\lnot \, \exists u \, R(v,u) \lor \lnot \, Q(x,v))) & \hspace{10mm} \mathrm{定理1.3の7)の右} \\
\sim & \exists x \, \exists v \, (\forall y \, (P(y) \supset Q(x,z)) \lor (\forall u \, \lnot \, R(v,u) \lor \lnot \, Q(x,v))) & \hspace{10mm} \mathrm{定理2.1の10)の右} \\
\sim & \exists x \, \exists v \, (\forall y \, (P(y) \supset Q(x,z)) \lor \forall u \, (\lnot \, R(v,u) \lor \lnot \, Q(x,v))) & \hspace{10mm} \mathrm{定理2.1の4)の左} \\
\sim & \exists x \, \exists v \, \forall u \, (\forall y \, (P(y) \supset Q(x,z)) \lor (\lnot \, R(v,u) \lor \lnot \, Q(x,v))) & \hspace{10mm} \mathrm{定理2.1の11)の左} \\
\sim & \exists x \, \exists v \, \forall u \, ((\lnot \, R(v,u) \lor \lnot \, Q(x,v)) \lor \forall y \, (P(y) \supset Q(x,z))) & \hspace{10mm} \mathrm{定理1.3の3)の右} \\
\sim & \exists x \, \exists v \, \forall u \, \forall y \, ((\lnot \, R(v,u) \lor \lnot \, Q(x,v)) \lor (P(y) \supset Q(x,z))) & \hspace{10mm} \mathrm{定理2.1の4)の左}
  \end{array}
 \end{equation*}
\end{enumerate}

\par
\vspace{5mm}

\par
\vspace{5mm}

%% 2.13

\noindent \textbf{問2.14} 
\par
\vspace{3mm}

\begin{enumerate}
  \renewcommand{\labelenumi}{\arabic{enumi}) }
\item
\begin{displaymath}
\setcounter{prfassumptioncounter}{0}
\prftree[r]{}
        {\prftree[r]{}
          {\prftree[r]{}
            {\prftree[r]{}
              {\prftree[r]{}
                {\prftree[r]{}
                  {\prftree[r]{}
                    { \, P(x) \to \, P(x) }
                    { \, P(x) \to \, P(x) , \,Q(x)}
                  }
                  { \to \, P(x) ,  P(x) \supset Q(x)}
                }
                { \to \forall x P(x) ,  P(x) \supset Q(x)}
              }
              { \to \forall x P(x) , \exists x ( P(x) \supset Q(x) )}
            }
            {\to \exists x ( P(x) \supset Q(x) ) , \forall x P(x)}
          }
          {\prftree[r]{}
            { \, Q(t) \to \, Q(t) }
            {\prftree[r]{}
              {\prftree[r]{}
                {\prftree[r]{}
                  { \,P(t) , \, Q(t) \to \, Q(t) )}
                  { \, Q(t) \to \, P(t) \supset Q(t) )}
                }
                { \, Q(t) \to \exists x ( P(x) \supset Q(x) )}
              }
              { \exists x Q(t) \to \exists x ( P(x) \supset Q(x) )}  
            }
          } 
          {\forall x P(x) \supset \exists Q(x) \to \exists x ( P(x) \supset Q(x) ) , \exists x ( P(x) \supset Q(x) )}
        }
        {\forall x P(x) \supset \exists Q(x) \to \exists x ( P(x) \supset Q(x) )}
\end{displaymath}\vspace{.2ex}
\item
誤り
\begin{displaymath}
%\setcounter{prfassumptioncounter}{0}
{\prftree[r]{}
  {\prftree[r]{}
    {\prftree[r]{}
      { \, P(x) \to \, P(x) , \,Q(x)}
      { \to \, P(x) ,  P(x) \supset Q(x)}
    }
    { \to \forall x P(x) ,  P(x) \supset Q(x)}
  }
  { \to \forall x P(x) , \exists x ( P(x) \supset Q(x) )}
}
\end{displaymath}
正しい
\begin{displaymath}
{\prftree[r]{}
  {\prftree[r]{}
    {\prftree[r]{}
      { \, P(x) \to \, P(x) , \,Q(x)}
      { \to \, P(x) ,  P(x) \supset Q(x)}
    }
    { \to \, P(x) , \exists x ( P(x) \supset Q(x) )}
    }
  { \to \forall x P(x) , \exists x ( P(x) \supset Q(x) )}
}
\end{displaymath}
\end{enumerate}

%% 3.1
%% 3.2

\appendix

\section{証明の書き方}
\begin{color}{blue}
\begin{itemize}
 \item 接続詞などに用いる用語を統一する(教科書を参考にする).
 \item 証明を書くときは,一行ずつ書いて改行する.
 \item サ変動詞を用いない.\sout{$\sim$として,$\sim$とする} $\ \Longrightarrow \ $ $\sim$と仮定する,$\sim$と置く,$\dots$となるような$\sim$をとる.
 \item 仮定が何で結論は何なのかを明示する.
 \item 問題文の情報を用いた場合は,問題文のどこを用いたのかを明示する.
 \item 推論する場合は,用いた根拠と用いた推論規則を明示する.
\end{itemize}
\end{color}


%% \begin{displaymath}
%% \setcounter{prfassumptioncounter}{0}
%% \prftree[r]{$\scriptstyle\supset\mathrm{I}$}
%%  {\prftree[r]{$\scriptstyle\land\mathrm{I}$}
  %% {\prftree[r]{$\scriptstyle\forall\mathrm{I}$}
  %%  {\prftree[r]{$\scriptstyle\land\mathrm{E1}$}
  %%   {\prftree[r]{$\scriptstyle\forall\mathrm{E}$}
  %%    {\prfboundedstyle=1\prfboundedassumption<assum:62>{\forall x \ (A \land B)}}
  %%    {A[z/x] \land B[z/x]}
  %%   }
  %%   {A[z/x]}
  %%  }
  %%  {\forall x\, A}
  %% }
  %% {\prftree[r]{$\scriptstyle\forall\mathrm{I}$}
  %%  {\prftree[r]{$\scriptstyle\land\mathrm{E2}$}
  %%   {\prftree[r]{$\scriptstyle\forall\mathrm{E}$}
  %%    {\prfboundedstyle=1\prfboundedassumption<assum:62>{\forall x \ (A \land B)}}
  %%    {A[z/x] \land B[z/x]}
  %%   }
  %%   {B[z/x]}
  %%  }
  %%  {\forall x\, B}
  %% }
%%   {\forall x \, A \land \forall x \, B}
%%  }
%% {\forall x \ (A \land B) \supset (\forall x \, A \land \forall x \, B)}
%% \end{displaymath}\vspace{.2ex}

\end{document}


%定義

%\documentclass[10pt,twocolumn]{jarticle} 
\documentclass[11pt,dvipdfmx]{jreport}

\usepackage{algorithmic, algorithm}
\usepackage{graphicx}
\usepackage{fancybox}
\usepackage{comment}
\usepackage{amsmath}
\usepackage{amssymb}
\usepackage{amsfonts}
\usepackage{euler}
%\usepackage{mathrsfs}
\usepackage{color}
\usepackage{ulem}
\usepackage{proof}
\usepackage[ND,SEQ,IMP]{prftree}
\usepackage[deluxe]{otf}


%\def\imp{\to}

%\pagestyle{empty}

%余白とか

%\setlength{\abovecaptionskip}{-10mm}
\setlength{\topmargin}{-3.0cm} 
\setlength{\textheight}{27.0cm} 
\setlength{\textwidth}{18.5cm}
\setlength{\oddsidemargin}{-1.3cm} 
\setlength{\columnsep}{.5cm}
\newcommand{\noin}{\noindent}
\catcode`@=\active \def@{\hspace{0.9bp}-\hspace{0.9bp}}
\newtheorem{dfn}{定義}[section]

%% \makeatletter
%% \renewcommand{\chapter}{\@startsection{chapter}{1}{\z@}
%% {1.5\Cvs \@plus.5\Cvs \@minus.2\Cvs}
%% {.5\Cvs \@plus.3\Cvs}
%% {\reset@font\Huge\mcfamily}}
%% \renewcommand{\section}{\@startsection{section}{1}{\z@}
%% {1.5\Cvs \@plus.5\Cvs \@minus.2\Cvs}
%% {.5\Cvs \@plus.3\Cvs}
%% {\reset@font\Large\mcfamily}}
%% \makeatother

%タイトル

\title{「情報科学における論理\footnote{小野寛晰, 日本評論社, 1994}」問題解答集(途中省略有り)}
\setcounter{footnote}{0}
\author{高田 篤司\thanks{神奈川大学理学部情報科学科} \and 原田 崇司\thanks{神奈川大学大学院理学研究科情報科学専攻 田中研究室}}
\date{\today}
\西暦

%タイトル作成

\begin{document}

\maketitle

\noindent \textbf{問1.1}の解答
\par
1) 正しくない.
\par
1) が正しくないことを証明する為には,$A \supset B$および$A$がともに充足可能であることを仮定して,$B$が充足可能であることを示せば良い.
\par
よって,初めに,
\begin{equation}
 \textrm{$A \supset B$および$A$がともに充足可能である}
 \label{eq:hypo1_1}
\end{equation}
と仮定する.そして,
\begin{equation}
 \textrm{論理式$A$を$p$,論理式$B$を$r \land \lnot r$}
 \label{eq:hypo1_2}
\end{equation}
と仮定する.
\par
仮定(\ref{eq:hypo1_2})より,論理式$A \supset B$,即ち,$p \supset r \land \lnot r$の真理値表は表\ref{tb:AimpB}となる.
\begin{table}[!htbp]
 \centering
   \caption{$p \supset r \land \lnot r \ \ (A \supset B)$の真理値表}
   \vspace{3mm}
   \begin{tabular}{c|c|c|c|c|c}
    $p$ & $r$ & $\lnot r$ & $p$ & $r \land \lnot r$ & $p \supset r \land \lnot r$ \\ \hline
    $t$ & $t$ & $f$       & $t$       & $f$                     & $f$ \\ \hline
    $t$ & $f$ & $t$       & $t$       & $f$                     & $f$ \\ \hline
    $f$ & $t$ & $f$       & $f$       & $f$                     & $t$ \\ \hline
    $f$ & $f$ & $f$       & $f$       & $f$                     & $t$ 
   \end{tabular}
   \label{tb:AimpB}
\end{table}

表\ref{tb:AimpB}より,$A \supset B$は充足可能である.
\par
さらに,表\ref{tb:AimpB}より,$A$は充足可能である.
\par
しかし,表\ref{tb:AimpB}より,$B$は充足可能でない.
\par
以上より,1)は正しくない.

\vspace{5mm}
2) 正しい.
\par
2) が正しいことを証明する為には,$A \supset B$がトートロジで$A$が充足可能であることを仮定して,$B$が充足可能であることを示せば良い.よって,初めに,
\begin{equation}
 \textrm{$A \supset B$がトートロジで$A$充足可能である}
 \label{eq:hypo1_3}
\end{equation}
と仮定する.
\par
仮定(\ref{eq:hypo1_3})より,$A \supset B$ がトートロジーで$A$が充足可能なので,$v(A) = t, \ v(A \supset B) = t$を満たす付値$v$が存在する.
\par
ここで,表\ref{tb:imply}より,$v(A)=t \ \land \ v(A \supset B) = t$ならば,$v(B) = t$である.
\begin{table}[!htbp]
 \centering
   \caption{$A \supset B$の真理値表}
   \vspace{3mm}
   \begin{tabular}{c|c|c}
    $A$ & $B$ & $A \supset B$ \\ \hline
    $t$ & $t$ & $t$ \\ \hline
    $t$ & $f$ & $f$ \\ \hline
    $f$ & $t$ & $t$ \\ \hline
    $f$ & $f$ & $t$ 
   \end{tabular}
   \label{tb:imply}
\end{table}


よって,$A \supset B$がトートロジーで$A$が充足可能なとき,$v(B) = t$となる付値$v$が存在するので,$B$も充足可能である.
\par
以上より,2)は正しい.

\newpage

\noindent \textbf{問2.14} 
\par
\vspace{3mm}

\begin{enumerate}
  \renewcommand{\labelenumi}{\arabic{enumi}) }
\item
\begin{displaymath}
\setcounter{prfassumptioncounter}{0}
\prftree[r]{}
        {\prftree[r]{}
          {\prftree[r]{}
            {\prftree[r]{}
              {\prftree[r]{}
                {\prftree[r]{}
                  {\prftree[r]{}
                    { \, P(x) \to \, P(x) }
                    { \, P(x) \to \, P(x) , \,Q(x)}
                  }
                  { \to \, P(x) ,  P(x) \supset Q(x)}
                }
                { \to \forall x P(x) ,  P(x) \supset Q(x)}
              }
              { \to \forall x P(x) , \exists x ( P(x) \supset Q(x) )}
            }
            {\to \exists x ( P(x) \supset Q(x) ) , \forall x P(x)}
          }
          {\prftree[r]{}
            { \, Q(t) \to \, Q(t) }
            {\prftree[r]{}
              {\prftree[r]{}
                {\prftree[r]{}
                  { \,P(t) , \, Q(t) \to \, Q(t) )}
                  { \, Q(t) \to \, P(t) \supset Q(t) )}
                }
                { \, Q(t) \to \exists x ( P(x) \supset Q(x) )}
              }
              { \exists x Q(t) \to \exists x ( P(x) \supset Q(x) )}  
            }
          } 
          {\forall x P(x) \supset \exists Q(x) \to \exists x ( P(x) \supset Q(x) ) , \exists x ( P(x) \supset Q(x) )}
        }
        {\forall x P(x) \supset \exists Q(x) \to \exists x ( P(x) \supset Q(x) )}
\end{displaymath}\vspace{.2ex}
\item
誤り
\begin{displaymath}
%\setcounter{prfassumptioncounter}{0}
{\prftree[r]{}
  {\prftree[r]{}
    {\prftree[r]{}
      { \, P(x) \to \, P(x) , \,Q(x)}
      { \to \, P(x) ,  P(x) \supset Q(x)}
    }
    { \to \forall x P(x) ,  P(x) \supset Q(x)}
  }
  { \to \forall x P(x) , \exists x ( P(x) \supset Q(x) )}
}
\end{displaymath}
正しい
\begin{displaymath}
{\prftree[r]{}
  {\prftree[r]{}
    {\prftree[r]{}
      { \, P(x) \to \, P(x) , \,Q(x)}
      { \to \, P(x) ,  P(x) \supset Q(x)}
    }
    { \to \, P(x) , \exists x ( P(x) \supset Q(x) )}
    }
  { \to \forall x P(x) , \exists x ( P(x) \supset Q(x) )}
}
\end{displaymath}
\end{enumerate}

\appendix

\section{証明の書き方}
\begin{color}{blue}
\begin{itemize}
 \item 接続詞などに用いる用語を統一する(教科書を参考にする).
 \item 証明を書くときは,一行ずつ書いて改行する.
 \item サ変動詞を用いない.\sout{$\sim$として,$\sim$とする} $\ \Longrightarrow \ $ $\sim$と仮定する,$\sim$と置く,$\dots$となるような$\sim$をとる.
 \item 仮定が何で結論は何なのかを明示する.
 \item 問題文の情報を用いた場合は,問題文のどこを用いたのかを明示する.
 \item 推論する場合は,用いた根拠と用いた推論規則を明示する.
\end{itemize}
\end{color}


%% \begin{displaymath}
%% \setcounter{prfassumptioncounter}{0}
%% \prftree[r]{$\scriptstyle\supset\mathrm{I}$}
%%  {\prftree[r]{$\scriptstyle\land\mathrm{I}$}
  %% {\prftree[r]{$\scriptstyle\forall\mathrm{I}$}
  %%  {\prftree[r]{$\scriptstyle\land\mathrm{E1}$}
  %%   {\prftree[r]{$\scriptstyle\forall\mathrm{E}$}
  %%    {\prfboundedstyle=1\prfboundedassumption<assum:62>{\forall x \ (A \land B)}}
  %%    {A[z/x] \land B[z/x]}
  %%   }
  %%   {A[z/x]}
  %%  }
  %%  {\forall x\, A}
  %% }
  %% {\prftree[r]{$\scriptstyle\forall\mathrm{I}$}
  %%  {\prftree[r]{$\scriptstyle\land\mathrm{E2}$}
  %%   {\prftree[r]{$\scriptstyle\forall\mathrm{E}$}
  %%    {\prfboundedstyle=1\prfboundedassumption<assum:62>{\forall x \ (A \land B)}}
  %%    {A[z/x] \land B[z/x]}
  %%   }
  %%   {B[z/x]}
  %%  }
  %%  {\forall x\, B}
  %% }
%%   {\forall x \, A \land \forall x \, B}
%%  }
%% {\forall x \ (A \land B) \supset (\forall x \, A \land \forall x \, B)}
%% \end{displaymath}\vspace{.2ex}

\end{document}

